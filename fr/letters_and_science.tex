% Sources :
%    + http://forum.mathematex.net/latex-f6/lettrine-encadree-t11998-20.html#p116343
%      francky gave me the table of the greek letters. Thank you !

%    + http://trucsmaths.free.fr/hist_symbol.htm
%      Significations of some symbolic notations.


\documentclass[fontsize=12pt]{scrartcl}
    \usepackage{template/general}
    \usepackage{template/math}
    \usepackage{specific}


    \usepackage[
        type={CC},
        modifier={by-nc-sa},
        version={4.0},
    ]{doclicense}


\begin{document}

% See ''Math mode'' of Herbert Voß

\abovedisplayskip=4pt plus 1pt minus 3pt
\abovedisplayshortskip=0pt plus 1pt
\belowdisplayskip=4pt plus 1pt minus 3pt
\belowdisplayshortskip=2.3pt plus 1pt minus 1.3pt


\addtitle{%
    title      = {Des lettres et des Sciences},
    name       = {Christophe BAL},
    mail       = {projetmbc@gmail.com},
    version    = {Version du 2017-11-02},
    addlicence = yes,
    repo       = {https://github.com/bc-writing/letters-and-science},
    repofolder = {fr}
}

\vspace{4cm}
\hrule
\setcounter{tocdepth}{2}
\tableofcontents

\vspace{1.5em}
\hrule
\newpage


\specialSubSection{Les lettres grecques}
    
    \input{content/greek_letters_table.tex}

    \newpage

    Nous présentons ci-après quelques situations illustrant l'utilisation des lettres grecques et de quelques autres en Mathématiques, en Sciences Physiques et en Chimie \emph{(ce document ne prétend pas être exhaustif)}.


\specialSubSection{Quelques exemples d'utilisation de lettres grecques}
    
    \input{content/greek_letters.tex}


\newpage
\specialSubSection{Des ensembles classiques}
    \input{content/sets.tex}


\newpage
\specialSubSection{Des constantes célèbres}
    %%%%%%%%%%%%%%%%%%%%%%%%%%%%%
%% DES CONSTANTES CELEBRES %%

%% PI %%
\cadre{$\pi$} Cette constante n'a plus besoin d'être présentée
	\footnote{Quoique... Le très bon livre de vulgarisation \oneBook{Le fascinant nombre $\pi$} de \oneAuthor{Jean-Paul}{Delahaye} contient beaucoup d'informations intéressantes.},
c'est le nombre réel, mais non rationnel, pi qui apparait par exemple dans $2 \pi R$ la longueur d'un cercle de rayon $R$, ou dans $\pi R^{2}$ l'aire d'un disque de rayon $R$, ou aussi dans $4 \pi R^{2}$ l'aire d'une sphère de rayon $R$, ou enfin dans $\frac{4 \pi R^3}{3}$ le volume d'une boule de rayon $R$. La lettre $\pi$ est le P minuscule grec qui fait référence ici au périmètre d'un cercle, plus rigoureusement à sa longueur. C'est à Leonhard Euler que l'on doit cette notation.

%% GAMMA %%
\cadre{$\gamma$} La constante gamma d'Euler-Mascheroni est une constante classique en Mathéma\-tiques. Elle est définie comme étant la limite suivante \emph{(dont on démontre l'existence et la finitude)} :
\begin{equation}
	\gamma = \lim_{n \rightarrow +\infty} \left( \sum_{k=1}^{n} \frac{1}{k} - \ln n \right)
\end{equation}

%% EXPONENTIELLE DE UN %%
\cadre{$\ee$} Par définition, $\ee =\exp 1$ est le nombre d'Euler qui est aussi nommé constante de Neper, ou base du logarithme népérien. Leonhard Euler détermina les deux formules suivantes \emph{(pour la notation $k!$, voir \vpageref{factoriel})} :
\begin{flalign}
	\ee  & = \sum_{k=0}^{+\infty} \frac{1}{k!} =  \lim_{n \rightarrow +\infty} \sum_{k=0}^{n} \frac{1}{k!}
	\\
	\ee  & = \lim_{n \rightarrow +\infty} \left( 1 + \frac{1}{n} \right)^n
\end{flalign}


%% NOMBRE COMPLEXE i %%
\cadre{$\ii$} L'ensemble $\CC$ des nombres complexes est constitué de nombres qui peuvent se mettre sous la forme $a + b \, \ii$ où $\ii^2 = -1$ et avec $a$ et $b$ des nombres réels. Cette notation est due à Leonhard Euler qui trouvait très jolie la formule $\exp \left( \ii \pi \right) = -1$ 
	\footnote{Étrange quand on obtient un résultat qui est juste $\ii^2$...}.



\newpage
\specialSubSection{Des lettres sens dessus dessous...}
    \input{content/up_and_down.tex}


\newpage
\specialSubSection{D'autre(s) lettre(s)...}
    \input{content/special.tex}


\newpage
\specialSubSection{Remerciements}
    \input{content/thanks.tex}


\specialSubSection{Journal de bord (chronologie inversée)}
    %%%%%%%%%%%%%%%%%%%%%
%% JOURNAL DE BORD %%

\begin{description}
	\item[2017-11-02] Ajout de la très jeune constante $\tau$ \emph{(merci à Vincent D. de me l'avoir fait connaître)}, et aussi de nouveaux exemples d'utilisation des lettres grecques $\Theta$, $\Psi$ et $\Omega$.
	
	Mise à jour de l'adresse pointant vers \mylink[GitHub]{https://github.com}.

	\item[2015-10-26] Mise en ligne du document et de ses sources sur \mylink[GitHub]{https://github.com}, et amélioration du code source \LaTeX.

	\item[2013-02-06] Ajout d'exemples d'utilisation en logique de la lettre T modifiée.

Ajout de la licence Creative Commons pour un usage non commercial.

	\item[2011-11-18] Mise en ligne de la première version.
\end{description}        

\end{document}
