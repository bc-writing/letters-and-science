%%%%%%%%%%%%%%%%%%%
%% REMERCIEMENTS %%

Ce document a été tapé en utilisant \LaTeX{}. De l'aide précieuse a été trouvée sur le forum \mylink[MathemaTeX]{http://forum.mathematex.net/} :

\begin{enumerate}
	\item La mise en forme à l'aide de lettrines encadrées utilise une solution technique qui se trouve ici : \mylink{forum.mathematex.net/latex-f6/lettrine-encadree-t11998.html}.

	\item La solution pour obtenir des notes de bas de page avec une numérotation sans parenthèse est dans ce message : \mylink{forum.mathematex.net/latex-f6/mise-en-forme-des-notes-de-bas-de-page-t12172.html}.

	\item Le code \LaTeX{} du tableau des lettres grecques provient de la discussion suivante : \mylink{forum.mathematex.net/latex-f6/lettrine-encadree-t11998-20.html}.

\end{enumerate}

\noindent Les origines indiquées pour certaines notations mathématiques se basent sur les explications données dans le site suivant : \mylink{www.trucsmaths.fr.st}.