%%%%%%%%%%%%%%%%%%%%%%%%%%%%%
%% DES CONSTANTES CELEBRES %%

%% PI %%
\cadre{$\pi$} Cette constante n'a plus besoin d'être présentée
	\footnote{Quoique... Le très bon livre de vulgarisation \oneBook{Le fascinant nombre $\pi$} de \oneAuthor{Jean-Paul}{Delahaye} contient beaucoup d'informations intéressantes.},
c'est le nombre réel, mais non rationnel, pi qui apparait par exemple dans $2 \pi R$ la longueur d'un cercle de rayon $R$, ou dans $\pi R^{2}$ l'aire d'un disque de rayon $R$, ou aussi dans $4 \pi R^{2}$ l'aire d'une sphère de rayon $R$, ou enfin dans $\frac{4 \pi R^3}{3}$ le volume d'une boule de rayon $R$. La lettre $\pi$ est le P minuscule grec qui fait référence ici au périmètre d'un cercle, plus rigoureusement à sa longueur. C'est à Leonhard Euler que l'on doit cette notation.

%% GAMMA %%
\cadre{$\gamma$} La constante gamma d'Euler-Mascheroni est une constante classique en Mathéma\-tiques. Elle est définie comme étant la limite suivante \emph{(dont on démontre l'existence et la finitude)} :
\begin{equation}
	\gamma = \lim_{n \rightarrow +\infty} \left( \sum_{k=1}^{n} \frac{1}{k} - \ln n \right)
\end{equation}

%% EXPONENTIELLE DE UN %%
\cadre{$\ee$} Par définition, $\ee =\exp 1$ est le nombre d'Euler qui est aussi nommé constante de Neper, ou base du logarithme népérien. Leonhard Euler détermina les deux formules suivantes \emph{(pour la notation $k!$, voir \vpageref{factoriel})} :
\begin{flalign}
	\ee  & = \sum_{k=0}^{+\infty} \frac{1}{k!} =  \lim_{n \rightarrow +\infty} \sum_{k=0}^{n} \frac{1}{k!}
	\\
	\ee  & = \lim_{n \rightarrow +\infty} \left( 1 + \frac{1}{n} \right)^n
\end{flalign}


%% NOMBRE COMPLEXE i %%
\cadre{$\ii$} L'ensemble $\CC$ des nombres complexes est constitué de nombres qui peuvent se mettre sous la forme $a + b \, \ii$ où $\ii^2 = -1$ et avec $a$ et $b$ des nombres réels. Cette notation est due à Leonhard Euler qui trouvait très jolie la formule $\exp \left( \ii \pi \right) = -1$ 
	\footnote{Étrange quand on obtient un résultat qui est juste $\ii^2$...}.
